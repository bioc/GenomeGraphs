%\VignetteIndexEntry{The GenomeGraphs users guide}
%\VignetteDepends{GenomeGraphs}
%\VignetteKeywords{Visualization}
%\VignettePackage{GenomeGraphs}
\documentclass[11pt]{article}
\usepackage{hyperref}
\usepackage{url}
\usepackage[authoryear,round]{natbib}
\bibliographystyle{plainnat}

\newcommand{\scscst}{\scriptscriptstyle}
\newcommand{\scst}{\scriptstyle}

\newcommand{\Rfunction}[1]{{\texttt{#1}}}
\newcommand{\Robject}[1]{{\texttt{#1}}}
\newcommand{\Rpackage}[1]{{\textit{#1}}}


\author{Steffen Durinck\footnote{steffen@stat.berkeley.edu} and James
  Bullard\footnote{bullard@stat.berkeley.edu}}
\usepackage{/Library/Frameworks/R.framework/Resources/share/texmf/Sweave}
\begin{document}
\title{The GenomeGraphs user's guide}

\maketitle

\tableofcontents
%%%%%%%%%%%%%%%%%%%%%%%%%%%%%%%%%%%%%%%%%%%%%%%%%%%%%%%%%%%%%%%%%%%%%%%%%%%%%%%
\section{Introduction}

Genomic data analyses can benifit from integrated visualization of the
genomic information.  The GenomeGraphs package uses the biomaRt
package to do live queries to Ensembl and translates
e.g. gene/transcript structures to viewports of the grid graphics
package, resulting in genomic information plotted together with your
data.  Possible genomics datasets that can be plotted are: Array CGH
data, gene expression data and sequencing data.

\begin{Schunk}
\begin{Sinput}
> library(GenomeGraphs)
\end{Sinput}
\end{Schunk}

\section{Creating a Ensembl annotation graphic}

To create an Ensembl annotation graphic, you need to decide what you
want to plot.  Genes and transcripts can be plotted individually using
the \Robject{Gene} and \Robject{Transcript} objects respectively.  Or
one can plot a gene region the forward strand or reverse strand only
or both.  In this section we will cover these different graphics.

\subsection{Plotting a Gene}

If one wants to plot annotation information from Ensembl then you need
to connect to the Ensembl BioMart database using the useMart function
of the biomaRt package.

\begin{Schunk}
\begin{Sinput}
> mart <- useMart("ensembl", dataset = "hsapiens_gene_ensembl")
\end{Sinput}
\begin{Soutput}
Checking attributes and filters ... ok
\end{Soutput}
\end{Schunk}

Next we can retrieve the gene structure of the gene of interest.

\begin{Schunk}
\begin{Sinput}
> gene <- new("Gene", id = "ENSG00000095203", 
+     type = "ensembl_gene_id", biomart = mart)
> gdPlot(gene)
\end{Sinput}
\end{Schunk}
\includegraphics{GenomeGraphs-004}

\subsection{Adding alternative transcripts}

To add alternative transcripts you first have to create a \Robject{Transcript} object.
Note that the order of the objects in the list determines the order in the plot.
\begin{Schunk}
\begin{Sinput}
> transcript <- new("Transcript", id = "ENSG00000095203", 
+     type = "ensembl_gene_id", biomart = mart)
> gdPlot(list(gene, transcript))
\end{Sinput}
\end{Schunk}
\includegraphics{GenomeGraphs-005}

\subsection{Plotting a gene region}

If you're interested in not just plotting one gene but a whole gene
region the you should create a \Robject{GeneRegion} object.  Note that
a \Robject{GeneRegion} object is strand specific.  In the example
below we will retrieve the genes on the forward (+) strand only and
add a genomic axis as well to give us the base positions.

\begin{Schunk}
\begin{Sinput}
> plusStrand <- new("GeneRegion", chromosome = "17", 
+     start = 30450000, end = 30550000, 
+     strand = "+", biomart = mart)
> genomeAxis <- makeGenomeAxis(add53 = TRUE)
> gdPlot(list(genomeAxis, plusStrand))
\end{Sinput}
\end{Schunk}
\includegraphics{GenomeGraphs-006}

Let's now add the genes on the negative strand as well and an ideogram
of chromosome 17, highlighting the region we are looking at.


\begin{Schunk}
\begin{Sinput}
> minStrand <- new("GeneRegion", chromosome = "17", 
+     start = 30450000, end = 30550000, 
+     strand = "-", biomart = mart)
> ideogram <- makeIdeogram(chromosome = 17)
> genomeAxis <- makeGenomeAxis(add53 = TRUE, 
+     add35 = TRUE)
> gdPlot(list(ideogram, plusStrand, genomeAxis, 
+     minStrand), minBase = 30450000, maxBase = 30550000)
\end{Sinput}
\end{Schunk}
\includegraphics{GenomeGraphs-007}

\section{Adding Array data to the plot}

\subsection{Array CGH and gene expression array data}
The \Robject{Generic Array} object enables plotting of expression and
CGH array data together with segments if available.  The array
intensity data should be given as a matrix, with in the rows t he
different probes and in the columns the different samples.  For each
probe the start location should be given using the probeStart
argument.  This should be a one column matrix.  Lets load some dummy
data.

\begin{Schunk}
\begin{Sinput}
> data("exampleData", package = "GenomeGraphs")
> minbase <- 180292097
> maxbase <- 180492096
> genesplus <- new("GeneRegion", start = minbase, 
+     end = maxbase, strand = "+", chromosome = "3", 
+     biomart = mart)
> genesmin <- new("GeneRegion", start = minbase, 
+     end = maxbase, strand = "-", chromosome = "3", 
+     biomart = mart)
> seg <- makeSegmentation(segStart, segEnd, 
+     segments, dp = DisplayPars(color = "black", 
+         lwd = 2, lty = "solid"))
> cop <- makeGenericArray(intensity = cn, 
+     probeStart = probestart, segmentation = seg, 
+     dp = DisplayPars(size = 3, color = "seagreen", 
+         type = "dot"))
> ideog <- makeIdeogram(chromosome = 3)
> expres <- makeGenericArray(intensity = intensity, 
+     probeStart = exonProbePos, dp = DisplayPars(color = "darkred", 
+         type = "point"))
> genomeAxis <- makeGenomeAxis(add53 = TRUE, 
+     add35 = TRUE)
> gdPlot(list(a = ideog, b = expres, c = cop, 
+     d = genesplus, e = genomeAxis, f = genesmin), 
+     minBase = minbase, maxBase = maxbase, 
+     labelCex = 2)
\end{Sinput}
\end{Schunk}
\includegraphics{GenomeGraphs-008}

\subsection{Exon array data}

The example below plots probe level exon array data and is usefull in
relating alternative splicing with known transcript structures.

\begin{Schunk}
\begin{Sinput}
> data("unrData", package = "GenomeGraphs")
> title <- makeTitle(text = "ENSG00000009307", 
+     color = "darkred")
> exon <- makeExonArray(intensity = unrData, 
+     probeStart = unrPositions[, 3], probeEnd = unrPositions[, 
+         4], probeId = as.character(unrPositions[, 
+         1]), nProbes = unrNProbes, dp = DisplayPars(color = "blue", 
+         mapColor = "dodgerblue2"), displayProbesets = FALSE)
> affyModel <- makeGeneModel(start = unrPositions[, 
+     3], end = unrPositions[, 4])
> gene <- new("Gene", id = "ENSG00000009307", 
+     biomart = mart)
> transcript <- new("Transcript", id = "ENSG00000009307", 
+     biomart = mart)
> legend <- makeLegend(c("affyModel", "gene"), 
+     fill = c("darkgreen", "orange"))
> gdPlot(list(title, exon, affyModel, gene, 
+     transcript, legend), minBase = min(exon@probeStart), 
+     maxBase = max(exon@probeEnd))
\end{Sinput}
\end{Schunk}
\includegraphics{GenomeGraphs-009}

\subsection{Plotting Conservation Data}
The UCSC genome browser offers downloadable conservation data for a
variety of species. Here we show how you can plot that conservation
data along with annotation. 

\begin{Schunk}
\begin{Sinput}
> yeastMart <- useMart("ensembl", dataset = "scerevisiae_gene_ensembl")
\end{Sinput}
\begin{Soutput}
Checking attributes and filters ... ok
\end{Soutput}
\begin{Sinput}
> minB <- 10000
> maxB <- 20000
> chrRoman <- as.character(as.roman(1))
> grP <- new("GeneRegion", start = minB, 
+     end = maxB, strand = "+", chromosome = chrRoman, 
+     biomart = yeastMart)
> grM <- new("GeneRegion", start = minB, 
+     end = maxB, strand = "-", chromosome = chrRoman, 
+     biomart = yeastMart)
> gaxis <- new("GenomeAxis", add53 = TRUE, 
+     add35 = TRUE)
> conserv <- yeastCons1[yeastCons1[, 1] > 
+     minB & yeastCons1[, 1] < maxB, ]
> consTrack <- new("BaseTrack", base = conserv[, 
+     1], value = conserv[, 2], dp = DisplayPars(lwd = 0.2, 
+     ylim = c(0, 1.25), color = "darkblue"))
> gdPlot(list(grP, gaxis, grM, conservation = consTrack))
\end{Sinput}
\end{Schunk}
\includegraphics{GenomeGraphs-010}


\section{Odds and Ends}
In addition to plotting the genes we can enable the plotting of names of genes. 

\begin{Schunk}
\begin{Sinput}
> plotGeneRegion <- function(chr = 1, minB = 9000, 
+     maxB = 13000, rot = 0, col = "green") {
+     chrRoman <- as.character(as.roman(1:17)[chr])
+     grP <- new("GeneRegion", start = minB, 
+         end = maxB, strand = "+", chromosome = chrRoman, 
+         biomart = yeastMart, dp = DisplayPars(plotId = TRUE, 
+             idRotation = rot, idColor = col))
+     gaxis <- new("GenomeAxis", add53 = TRUE, 
+         add35 = TRUE, littleTicks = FALSE)
+     gdPlot(list(grP, gaxis), minBase = minB, 
+         maxBase = maxB)
+ }
> plotGeneRegion(col = "yellow", rot = 90)
\end{Sinput}
\end{Schunk}
\includegraphics{GenomeGraphs-011}

Finally, if you are interested in seeing how things look you can just
plot the object without the list, or without the \emph{minBase}, \emph{maxBase}
arguments. 
\begin{Schunk}
\begin{Sinput}
> gdPlot(new("GeneRegion", start = 9000, 
+     end = 15000, biomart = yeastMart, 
+     strand = "-", chromosome = "I", dp = DisplayPars(plotId = TRUE)))
\end{Sinput}
\end{Schunk}
\includegraphics{GenomeGraphs-012}

\subsection{Overlays} 
\texttt{Overlays} can be used to annotate different regions of the
plot. Currently, we can draw boxes and write text on the plot.

\begin{Schunk}
\begin{Sinput}
> ga <- new("GenomeAxis")
> grF <- new("GeneRegion", start = 10000, 
+     end = 20000, chromosome = "I", strand = "+", 
+     biomart = yeastMart)
> grR <- new("GeneRegion", start = 10000, 
+     end = 20000, chromosome = "I", strand = "-", 
+     biomart = yeastMart)
> bt <- new("BaseTrack", base = yeastCons1[, 
+     1], value = yeastCons1[, 2])
> hr1 <- makeRectangleOverlay(start = 11000, 
+     end = 13000)
> hr2 <- makeRectangleOverlay(start = 15900, 
+     end = 16500)
> gdPlot(list(grF, ga, grR, bt), overlays = list(hr1, 
+     hr2))
\end{Sinput}
\end{Schunk}
\includegraphics{GenomeGraphs-013}

A little nifty feature is to allow alpha blending to make things
slightly transparent. If the device you wish to plot on however, does
not support transparency then you will get a warning. 

\begin{Schunk}
\begin{Sinput}
> ro <- makeRectangleOverlay(start = 11000, 
+     end = 13000, region = c(1, 3), dp = DisplayPars(color = "green", 
+         alpha = 0.3))
> to <- makeTextOverlay("here is some text", 
+     xpos = 15000, ypos = 0.95)
> gdPlot(list(grF, ga, grR, bt), overlay = c(ro, 
+     to))
\end{Sinput}
\end{Schunk}
\includegraphics{GenomeGraphs-014}

Also, one can use "absolute" coordinates to specify a region just in
case one wants to be a bit more precise. 
\begin{Schunk}
\begin{Sinput}
> roR <- makeRectangleOverlay(start = 0.1, 
+     end = 0.3, coords = "absolute", dp = DisplayPars(fill = "grey", 
+         alpha = 0.2, lty = "dashed"), 
+     region = c(0.4, 0.7))
> gdPlot(list(grF, ga, grR, bt), overlays = list(ro, 
+     roR))
\end{Sinput}
\end{Schunk}
\includegraphics{GenomeGraphs-015}

\subsection{GenomeGraphs Classes}
\begin{table}[bp!]
    \begin{center}
      \begin{tabular}{@{}cp{8cm}@{}}
        \hline
        class & description \\
        \hline
        \texttt{gdObject} &  the root class of the system, never directly instantiated \\
        \texttt{Gene} &  class representing a gene \\
        \texttt{GeneRegion} & class defining a region of a
        chromsome, generally a set of genetic elements (genes) \\
        \texttt{Transcript} & class defining a transcript \\
        \texttt{TranscriptRegion} &  class defining a region of a chromsome, generally a set of genetic elements (transcripts)\\
        \texttt{Ideogram} &  an ideogram \\
        \texttt{Title} &  class to draw a title  \\
        \texttt{Legend} &  class to draw a legend  \\
        \texttt{GenomeAxis} &  class to draw a axis \\
        \texttt{Segmentation} &  class to draw horizontal lines in various sets of data \\
        \texttt{GenericArray} &  class to draw data from microarrays. \\
        \texttt{ExonArray} &  class to draw data from exon microarrays. \\
        \texttt{GeneModel} & class to draw custom gene models (intron-exon structures)  \\
        \texttt{BaseTrack} &  class to draw whatever kind of data at a given base \\
        \texttt{MappedRead} &  class to plot sequencing reads that are mapped to the genome \\
        \texttt{DisplayPars} & class managing various plotting parameters \\
        \texttt{AnnotationTrack} & class used to represent custom annotation \\
        \texttt{Overlay} & root class for overlays, never directly instantiated \\
        \texttt{RectangleOverlay} & class to represent rectangular regions of interest \\
        \texttt{TextOverlay} & class to draw text on plots \\
        \hline
      \end{tabular}
    \end{center}
  \end{table}

\begin{Schunk}
\begin{Sinput}
> data("seqDataEx", package = "GenomeGraphs")
> str = seqDataEx$david[, "strand"] == 1
> biomart = useMart("ensembl", "scerevisiae_gene_ensembl")
\end{Sinput}
\begin{Soutput}
Checking attributes and filters ... ok
\end{Soutput}
\begin{Sinput}
> pList = list("-" = new("GeneRegion", chromosome = "IV", 
+     start = 1300000, end = 1310000, strand = "-", 
+     biomart = biomart, dp = DisplayPars(plotId = TRUE, 
+         idRotation = 0, cex = 0.5)), new("GenomeAxis", 
+     dp = DisplayPars(byValue = 1e+05, 
+         size = 3)), "+" = new("GeneRegion", 
+     chromosome = "IV", start = 1300000, 
+     end = 1310000, strand = "+", biomart = biomart, 
+     dp = DisplayPars(plotId = TRUE, idRotation = 0, 
+         cex = 0.5)), Nagalakshmi = new("BaseTrack", 
+     base = seqDataEx$snyder[, "location"], 
+     value = seqDataEx$snyder[, "counts"], 
+     dp = DisplayPars(lwd = 0.3, color = "darkblue", 
+         ylim = c(0, 300))), "David +" = new("GenericArray", 
+     probeStart = seqDataEx$david[str, 
+         "location"], intensity = seqDataEx$david[str, 
+         "expr", drop = FALSE], dp = DisplayPars(pointSize = 0.5)), 
+     "David -" = new("GenericArray", probeStart = seqDataEx$david[!str, 
+         "location"], intensity = seqDataEx$david[!str, 
+         "expr", drop = FALSE], dp = DisplayPars(color = "darkgreen", 
+         pointSize = 0.5)), Lee = new("BaseTrack", 
+         base = seqDataEx$nislow[, "location"], 
+         value = seqDataEx$nislow[, "evalue"], 
+         dp = DisplayPars(color = "grey", 
+             lwd = 0.25)), Conservation = new("BaseTrack", 
+         base = seqDataEx$conservation[, 
+             "location"], value = seqDataEx$conservation[, 
+             "score"], dp = DisplayPars(color = "gold4", 
+             lwd = 0.25)))
> gdPlot(pList, minBase = 1301500, maxBase = 1302500, 
+     overlay = makeRectangleOverlay(start = 1302105, 
+         end = 1302190, region = c(4, 8), 
+         dp = DisplayPars(alpha = 0.2)))
\end{Sinput}
\end{Schunk}
\includegraphics{GenomeGraphs-016}

\end{document}
